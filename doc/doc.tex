\documentclass[a4paper,12pt]{article}

% PACKAGES
\usepackage[utf8]{inputenc}
\usepackage[brazil]{babel}
% math packages
\usepackage{amsmath}
\usepackage{amstext}

\usepackage{url}

\usepackage{multicol}
\usepackage{graphicx}

%
\graphicspath{ {./img/} }
\setlength{\parskip}{1.3ex plus 0.5ex minus 0.3ex}

% HEADER
\title{Trabalho Prático 1 - Estacionamento}
\author{
    Arthur Xavier\\
    \texttt{xavier@dcc.ufmg.br}
    \and
    Alexandre Pretti\\
    \texttt{email@dcc.ufmg.br}
    \and
    Vitor\\
    \texttt{email@dcc.ufmg.br}
}
\date{13 de Abril de 2016}

% DOCUMENT
\begin{document}

% TITLE
\maketitle

%%%%%%%%%%%%%%%%%%%%%%%%%%%%%%%%%%%%%%%%%%%%%%%%%%%%%%%%%%%%%%%%%%%%%%%%%%%%%%%
% INTRODUÇÃO
%%%%%%%%%%%%%%%%%%%%%%%%%%%%%%%%%%%%%%%%%%%%%%%%%%%%%%%%%%%%%%%%%%%%%%%%%%%%%%%
\section{Introdução}

Este trabalho consiste na implementação de um sistema de alocação de vagas para um estacionamento, visando a automatização do processo de entrada e saída do estabelecimento.

O estacionamento em questão possui quatro níveis e cada nível possui dez vagas, divididas entre motocicletas, veículos pequenos e grandes e vagas para portadores de necessidades especiais.

Ao chegar ao estacionamento, o cliente acessa um totem onde ele informa o tipo do seu carro. O sistema deve então informar se existe ou não vaga disponível que ele pode utilizar, em caso afirmativo, ele deve informar em qual nível e posição essa vaga fica, caso contrário, deve informar LOTADO. As vagas são numeradas sequencialmente de acordo com o nível e tipo de vaga.

Na saída do estacionamento, o cliente deve receber um ticket com o valor que deve ser pago. O custo é calculado de acordo com um valor associado ao tipo da vaga utilizada e à quantidade de tempo.

\newpage

%%%%%%%%%%%%%%%%%%%%%%%%%%%%%%%%%%%%%%%%%%%%%%%%%%%%%%%%%%%%%%%%%%%%%%%%%%%%%%%
% Implementação
%%%%%%%%%%%%%%%%%%%%%%%%%%%%%%%%%%%%%%%%%%%%%%%%%%%%%%%%%%%%%%%%%%%%%%%%%%%%%%%
\section{Implementação}

% Algoritmo de alocação de veículos em vagas sequencial. Os veículos são estacionados sequencialmente em vagas, de forma a escolher a primeira vaga possível, não verificando se haveriam vagas próximas melhores.


%%%%%%%%%%%%%%%%%%%%%%%%%%%%%%%%%%%%%%%%%%%%%%%%%%%%%%%%%%%%%%%%%%%%%%%%%%%%%%%
% TESTES
%%%%%%%%%%%%%%%%%%%%%%%%%%%%%%%%%%%%%%%%%%%%%%%%%%%%%%%%%%%%%%%%%%%%%%%%%%%%%%%
\section{Testes}

Foram criadas três rotinas de teste de aceitação automatizadas baseadas nos critérios especificados pelo problema. As três rotinas de teste criadas visam testar diferentes casos que podem ocorrer durante a execução do programa. O processo de teste implementado possui um arquivo de entrada e um com a saída esperada supondo o correto funcionamento do programa. O programa é, então, executado sobre a entrada fornecida e a saída comparada com o arquivo contendo a saída esperada. Caso haja alguma divergência, o teste falha.

Os três testes implementados foram:
\begin{enumerate}
    \item \textbf{Normal -} testa o funcionamento do programa em condições normais, onde nenhum erro ou comportamento tratável ocorre;
    \item \textbf{Lotado -} testa o comportamento e a reação do programa ao verificar que o estacionamento encontra-se em capacidade máxima e um veículo tenta estacionar;
    \item \textbf{Inválido -} testa o comportamento do programa quando de uma entrada inválida.
\end{enumerate}

Os testes possuem cobertura suficiente para atestar o correto funcionamento do programa de acordo com as especificações e suposições.

%%%%%%%%%%%%%%%%%%%%%%%%%%%%%%%%%%%%%%%%%%%%%%%%%%%%%%%%%%%%%%%%%%%%%%%%%%%%%%%
% CONCLUSÃO
%%%%%%%%%%%%%%%%%%%%%%%%%%%%%%%%%%%%%%%%%%%%%%%%%%%%%%%%%%%%%%%%%%%%%%%%%%%%%%%
\section{Conclusão}

O objetivo deste trabalho relativo à disciplina de Programação Modular é buscar compreender e aplicar os conceitos aprendidos em sala de aula quanto a \emph{Orientação a Objetos} e linguagem \emph{Java}. Aplicar estes conceitos em exemplos reais é o que completa o aprendizado.

As dificuldades de compreensão do problema, como por exemplo qual algoritmo de alocação de veículos em vagas utilizar foram resolvidas por meio de suposições a fim de completar as lacunas cognitivas abertas pela descrição do problema. Acerca da implementação, as principais dificuldades se deram na importação e uso de bibliotecas padrão da JDK 7.

O resultado, entretanto, foi além de satisfatório, confirmado pelos testes executados e pela qualidade do código.

%%%%%%%%%%%%%%%%%%%%%%%%%%%%%%%%%%%%%%%%%%%%%%%%%%%%%%%%%%%%%%%%%%%%%%%%%%%%%%%
% BIBLIOGRAFIA
%%%%%%%%%%%%%%%%%%%%%%%%%%%%%%%%%%%%%%%%%%%%%%%%%%%%%%%%%%%%%%%%%%%%%%%%%%%%%%%
\section{Bibliografia}

\begin{enumerate}
    \item Oracle. \emph{Java™ Platform, Standard Edition 7 API Specification.} Abr 2016. \url{https://docs.oracle.com/javase/7/docs/api/}
\end{enumerate}

\end{document}
